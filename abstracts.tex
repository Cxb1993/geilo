\documentclass[a4paper,10pt]{article}
\usepackage[utf8]{inputenc}
\usepackage{amsmath, pdfpages, pdflscape, lscape, color, listings, hyperref, amssymb,graphicx,textcomp,varioref,afterpage,subcaption, float} 

\setlength\topmargin{-0,5in}
\setlength\textheight{9in}

\usepackage[margin=2.5cm]{geometry}

\newcommand{\Fig}[1]{Figure \ref{#1}}
\newcommand{\fig}[1]{figure \ref{#1}}
\newcommand{\tab}[1]{table \ref{#1}}
\newcommand{\eq}[1]{equation \ref{#1}}
\newcommand{\Eq}[1]{Equation \ref{#1}}
\newcommand{\alg}[1]{algorithm \ref{#1}}
\newcommand{\Alg}[1]{Algorithm \ref{#1}}
\newcommand{\chp}[1]{chapter  \ref{#1}}
\newcommand{\Chp}[1]{Chapter  \ref{#1}}
\newcommand{\e}[1]{\cdot 10^{#1}}
\newcommand{\h}{\hbar}
\newcommand{\der}[2]{\frac{\partial #1}{\partial #2}}
\newcommand{\dder}[2]{\frac{\partial^2 #1}{\partial #2^2}}

%opening
\title{Abstracts Geilo Winter School}
\author{Simen Tennøe and Jonathan Feinberg}

\begin{document}
\maketitle
\newpage


% tydligere titler med PCE in front
\section{Polynomial chaos expansion part 1: Method Introduction}

Polynomial chaos expansion is a class of methods for determining the
uncertainties in forward models given uncertainty in input parameters.
% Mangler en setning om hvorfor PCE har noe verdi (svar: den er raskere enn andre metoder)
In this first lecture we will give an introduction to the general
concepts necessary to understand the foundation of the thoery.
Along the lecture will also give an introduction to a software
package called Chaospy, a Python toolbox developed specifically to
use polynomial chaos expansion in practice.



\section{Practice of polynomial chaos expansion}
One important set of methods to get the coefficients of the
polynomial chaos expansion are collocation methods.
These are methods that require the residue of the governing
equations to be zero at specific points in the computations.
We will take a closer look at two subclasses of this set of
methods, the pseudospectral approach utilizing sparse grids and a
interpolation approach.  After this lecture you will be able to do
a polynomial chaos expansion using Chaospy on simpler problems.

% Pseudo-spectral methods collocation



\section{Advanced topics in polynomial chaos}
This lecture covers more advanced topics, such as what happens when
you no longer have independent variables.
This problem is solved by using a Rosenblatt transformation on the
system parameters.
%Second to last we generalize the polynomial chaos expansion by using different orthogonal polynomials.
We will end with a walkthrough of the Galerkin method, another
often used method for finding the coefficients of a generalized
polynomial chaos expansion, again using Chaospy.



\end{document}
