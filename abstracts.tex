\documentclass[a4paper,10pt]{article}
\usepackage[utf8]{inputenc}
\usepackage{amsmath, pdfpages, pdflscape, lscape, color, listings, hyperref, amssymb,graphicx,textcomp,varioref,afterpage,subcaption, float} 

\setlength\topmargin{-0,5in}
\setlength\textheight{9in}

\usepackage[margin=2.5cm]{geometry}

\newcommand{\Fig}[1]{Figure \ref{#1}}
\newcommand{\fig}[1]{figure \ref{#1}}
\newcommand{\tab}[1]{table \ref{#1}}
\newcommand{\eq}[1]{equation \ref{#1}}
\newcommand{\Eq}[1]{Equation \ref{#1}}
\newcommand{\alg}[1]{algorithm \ref{#1}}
\newcommand{\Alg}[1]{Algorithm \ref{#1}}
\newcommand{\chp}[1]{chapter  \ref{#1}}
\newcommand{\Chp}[1]{Chapter  \ref{#1}}
\newcommand{\e}[1]{\cdot 10^{#1}}
\newcommand{\h}{\hbar}
\newcommand{\der}[2]{\frac{\partial #1}{\partial #2}}
\newcommand{\dder}[2]{\frac{\partial^2 #1}{\partial #2^2}}

%opening
\title{Abstracts Geilo Winter School}
\author{Simen Tennøe and Jonathan Feinberg}

\begin{document}
\maketitle
\newpage


\section{Polynomial chaos expansions part I: Method Introduction}

Polynomial chaos expansion is a class of methods for determining the
uncertainties in forward models given uncertainty in the input parameters.
It is a new method made popular in recent years because of
its very fast convergence properties.
In this first lecture we will introduce the
general concepts necessary to understand the foundation of the theory.
In parallel we will also give an introduction to Chaospy: a Python
toolbox specifically developed to implement polynomial chaos
expansions.

\section{Polynomial chaos expansions part II: Practical Implementation}

Implementing polynomial chaos expansions in practice comes with a
few challenges.
In this lecture we will teach how to address these challenges.
In particular we will be looking at non-intrusive methods where we
can assume that the underlying model solver is a ``black box''.
We will show how to use Chaospy on a large collection of problems
with a walk-through of initial setup, implementation and analysis.

\section{Polynomial chaos expansions part III: Some advanced topics}

In some cases the basic theory of polynomial chaos expansion is not
enough.
For example, one fundamental assumption in the theory is that the
input parameters are stochastically independent.
In this lecture we will take a look at what
can be done when this assumption does not hold.
In addition we will take a look at intrusive polynomial chaos
methods.
Here we couple the polynomial chaos expansion directly to
a set of governing equations, resulting in a higher accuracy in
estimation.

\end{document}
