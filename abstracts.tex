\documentclass[a4paper,10pt]{article}
\usepackage[utf8]{inputenc}

\begin{document}

\section{Polynomial chaos expansions part I: Method Introduction}

Polynomial chaos expansion is a class of methods for determining the
uncertainties in forward models given uncertainty in the input parameters.
It is a new method made popular in recent years because of
its very fast convergence.
In this first lecture we will introduce the
general concepts necessary to understand the foundation of the theory.
We will in parallel introduce Chaospy: a Python
toolbox specifically developed to implement polynomial chaos
expansions.

\section{Polynomial chaos expansions part II: Practical Implementation}

Implementing polynomial chaos expansions in practice comes with a
few challenges. This lecture will teach how to address these.
In particular we will be looking at non-intrusive methods where we
can assume that the underlying model solver is a "black box".
We will show how to use Chaospy on a large collection of problems,
with a walk-through of the initial setup, implementation and analysis.

\section{Polynomial chaos expansions part III: Some advanced topics}

In some cases the basic theory of polynomial chaos expansion is not
enough.
For example, one fundamental assumption is that the
input parameters are stochastically independent.
In this lecture we will take a look at what
can be done when this assumption no longer holds.
Additionaly we will look at intrusive polynomial chaos
methods.
Here we couple the polynomial chaos expansion directly to
the set of governing equations, which results in a higher accuracy in
the estimations.

\end{document}
