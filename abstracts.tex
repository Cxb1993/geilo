\documentclass[a4paper,10pt]{article}
\usepackage[utf8]{inputenc}
\usepackage{amsmath, pdfpages, pdflscape, lscape, color, listings, hyperref, amssymb,graphicx,textcomp,varioref,afterpage,subcaption, float} 

\setlength\topmargin{-0,5in}
\setlength\textheight{9in}

\usepackage[margin=2.5cm]{geometry}

\newcommand{\Fig}[1]{Figure \ref{#1}}
\newcommand{\fig}[1]{figure \ref{#1}}
\newcommand{\tab}[1]{table \ref{#1}}
\newcommand{\eq}[1]{equation \ref{#1}}
\newcommand{\Eq}[1]{Equation \ref{#1}}
\newcommand{\alg}[1]{algorithm \ref{#1}}
\newcommand{\Alg}[1]{Algorithm \ref{#1}}
\newcommand{\chp}[1]{chapter  \ref{#1}}
\newcommand{\Chp}[1]{Chapter  \ref{#1}}
\newcommand{\e}[1]{\cdot 10^{#1}}
\newcommand{\h}{\hbar}
\newcommand{\der}[2]{\frac{\partial #1}{\partial #2}}
\newcommand{\dder}[2]{\frac{\partial^2 #1}{\partial #2^2}}

%opening
\title{Abstracts Geilo Winter School}
\author{Simen Tennøe}

\begin{document}
\maketitle
\newpage


\section{Introduction to polynomial chaos expansions}
Polynomial chaos expansion is a method for determining the propagation of uncertainties given uncertainties in the system parameters. This lecture intends to give an introduction to the general concepts necesary to understand the foundation of polynomial chaos expansion. It also introduces the fundamental ideas behind polynomial chaos, along with examples on how to use Chaospy, a python package developed specificaly to do polynomial chaos expansion.




\section{Practice of polynomial chaos expansion}
One important set of methods to get the coefficients of the polynomial chaos expansion are collocation methods. These are methods that require the residue of the governing equations to be zero at specific points in the computations. We will take a closer look at two subclasses of this set of methods, the pseudospectral approach utilizing sparse grids and a interpolation approach.
After this lecture you will be able to do a polynomial chaos expansion using chaospy on simpler problems.

% Pseudo-spectral methods
% collocation



\section{Advanced topics in polynomial chaos}
This lecture covers more advanced topics, such as what happens when you no longer have independent variables. Another topic is if the uncertain parameters have a non-Gaussian distribution, which is solved by using a Rosenblatt transformation on the parameters. %Second to last we generalize the polynomial chaos expansion by using different orthogonal polynomials.
We will end with a walkthrough of the Galerkin method, another often used method for finding the coefficients of a generalized polynomial chaos expansion, again using chaospy.



\end{document}
