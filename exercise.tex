\documentclass{beamer}
\usepackage[utf8]{inputenc}
\usepackage{amsmath, pdfpages, pdflscape, lscape, color, listings, hyperref, amssymb, graphicx,textcomp,varioref, afterpage, subcaption, float, bm, tikz, multicol} 

\global
\newcommand{\Fig}[1]{Figure \ref{#1}}
\newcommand{\fig}[1]{figure \ref{#1}}
\newcommand{\tab}[1]{table \ref{#1}}
\newcommand{\eq}[1]{equation \ref{#1}}
\newcommand{\Eq}[1]{Equation \ref{#1}}
\newcommand{\alg}[1]{algorithm \ref{#1}}
\newcommand{\Alg}[1]{Algorithm \ref{#1}}
\newcommand{\chp}[1]{chapter  \ref{#1}}
\newcommand{\Chp}[1]{Chapter  \ref{#1}}
\newcommand{\e}[1]{\cdot 10^{#1}}
\newcommand{\h}{\hbar}
\newcommand{\der}[2]{\frac{\partial #1}{\partial #2}}
\newcommand{\dder}[2]{\frac{\partial^2 #1}{\partial #2^2}}
\newcommand{\p}{\boldsymbol{P}}
\newcommand{\q}{\boldsymbol{q}}
\newcommand{\norm}[1]{\left\lVert#1\right\rVert}
\newcommand{\coef}[2]{\frac{\langle #1,#2\rangle_Q}{\norm{#2}^}}


\newenvironment{test}[1]
{
 \usebackgroundtemplate{}
 \color{gray!30!black}
   \begin{tikzpicture}[remember picture, overlay]
     \node[anchor = center, opacity=.25] (image) at (current page.center) {\includegraphics[scale=0.25]{chaospy_logo.jpg}};
   \end{tikzpicture}
 \begin{frame}[fragile,enviroment=chaospy]
   
}
{
 \end{frame}
}

\lstset{
escapeinside=||
}


\newenvironment{chaospy}[1]
{\color{gray!30!black}
     \color{gray!30!black}
     \usebackgroundtemplate{
   \begin{tikzpicture}[remember picture, overlay]
     \node[anchor = center, opacity=.25] (image) at (current page.center) {\includegraphics[scale=0.25]{chaospy_logo.jpg}};
   \end{tikzpicture}}
     \begin{frame}[fragile,environment=chaospy]
    \frametitle{{#1}}}
{\end{frame}}


\definecolor{keywords}{RGB}{255,0,90}
\definecolor{comments}{RGB}{0,0,113}
\definecolor{red}{RGB}{160,0,0}
\definecolor{green}{RGB}{0,150,0}
 
\usetheme{kalkulo}

\graphicspath{{./figures/}}


\title{Polynomial chaos expansions: Exercises}
\author{Jonathan Feinberg and Simen Tennøe}


\begin{document}



\begin{frame}
  \maketitle
\end{frame}



\begin{chaospy}{Code example for Gaussian quadrature}
    \scriptsize
\begin{lstlisting}[language=python]
dist = cp.Normal()
nodes, weights = cp.generate_quadrature(2, dist, rule="G")
print nodes
[[-1.73205081  0.          1.73205081]]
print weights
[ 0.16666667  0.66666667  0.16666667]
\end{lstlisting}
\begin{lstlisting}[language=python]
dist = cp.Uniform()
nodes, weights = cp.generate_quadrature(2, dist, rule="G")
print nodes
[[ 0.11270167  0.5         0.88729833]]
print weights
[ 0.27777778  0.44444444  0.27777778]
\end{lstlisting}
\end{chaospy}


\begin{frame}
\frametitle{Traveling with constant velocity, a simple problem}
 Distance when traveling at constant velocity is
\[s(t) = vt\]
\pause
where the velocity is measured to
\[v \sim \text{Normal(5, 0.1)}\]
\pause
\begin{alert}{Task:}
 Find the expectation value and variance in the time interval $t=[0,1]$  and plot them, use Gaussian Quadrature.
\end{alert}

\end{frame}

\begin{frame}
\frametitle{Traveling with constant acceleration}
 Distance when traveling with constant acceleration is
 \[s(t) = v_0t + \frac{1}{2}at^2\]
\pause
The initial velocity and acceleration is measured to
\[v_0 \sim \text{Uniform(1, 2)} \qquad a \sim \text{Normal(2, 0.5)}\]
\pause
\begin{alert}{Task:}
Find the expectation value and variance in the time interval $t=[0,1]$  and plot them, use Gaussian Quadrature.
\end{alert}
\end{frame}


\begin{frame}
\frametitle{Traveling with constant acceleration, different rules}
 Distance when traveling with constant acceleration is
 \[s(t) = v_0t + \frac{1}{2}at^2\]
The initial velocity and acceleration is measured to
\[v_0 \sim \text{Uniform(1, 2)} \qquad a \sim \text{Normal(2, 0.5)}\]
\begin{alert}{Task:}
 Find the expectation value and variance in the time interval $t=[0,1]$ and plot them. Do this for 
 \begin{itemize}
  \item Quadrature rule: Gaussian quadrature
  \item Quadrature rule: Clenshaw Curtis
  \item Point collocation rule: Least Square
 \end{itemize}
\end{alert}
\end{frame}

\begin{frame}
\frametitle{Traveling with constant acceleration, sensitivity}
 Distance when traveling with constant acceleration is
 \[s(t) = v_0t + \frac{1}{2}at^2\]
The initial velocity and acceleration is measured to
\[v_0 \sim \text{Uniform(1, 2)} \qquad a \sim \text{Normal(2, 0.5)}\]
\begin{alert}{Task:}
 Find the sensitivity in the time interval $t=[0,1]$.
\end{alert}
\end{frame}

\begin{frame}
\frametitle{Traveling with constant acceleration, convergence}
 Distance when traveling with constant acceleration is
 \[s(t) = v_0t + \frac{1}{2}at^2\]
The initial velocity and acceleration is measured to
\[v_0 \sim \text{Uniform(1, 2)} \qquad a \sim \text{Normal(2, 0.5)}\]
\begin{alert}{Task:}
 Calculate the convergence of the expectation value and variance in the interval $t=[0,1]$.
\end{alert}
\end{frame}
\
  
\end{document}
