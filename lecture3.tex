\documentclass{beamer}
\usepackage[utf8]{inputenc}
\usepackage{amsmath, pdfpages, pdflscape, lscape, color, listings, hyperref, amssymb, graphicx,textcomp,varioref, afterpage, subcaption, float, bm, tikz, multicol} 

\global
\newcommand{\Fig}[1]{Figure \ref{#1}}
\newcommand{\fig}[1]{figure \ref{#1}}
\newcommand{\tab}[1]{table \ref{#1}}
\newcommand{\eq}[1]{equation \ref{#1}}
\newcommand{\Eq}[1]{Equation \ref{#1}}
\newcommand{\alg}[1]{algorithm \ref{#1}}
\newcommand{\Alg}[1]{Algorithm \ref{#1}}
\newcommand{\chp}[1]{chapter  \ref{#1}}
\newcommand{\Chp}[1]{Chapter  \ref{#1}}
\newcommand{\e}[1]{\cdot 10^{#1}}
\newcommand{\h}{\hbar}
\newcommand{\der}[2]{\frac{\partial #1}{\partial #2}}
\newcommand{\dder}[2]{\frac{\partial^2 #1}{\partial #2^2}}
\newcommand{\p}{\boldsymbol{P}}
\newcommand{\q}{\boldsymbol{q}}
\newcommand{\norm}[1]{\left\lVert#1\right\rVert}
\newcommand{\coef}[2]{\frac{\langle #1,#2\rangle_{\!Q}}{\norm{#2}^}}
\newcommand{\inner}[1]{\left\langle #1 \right\rangle_{\!Q}}

\newcommand{\E}[1]{\mbox{E}\!\left( #1 \right)}
\newcommand{\Var}[1]{\mbox{Var}\!\left( #1 \right)}

\newenvironment{test}[1]
{
 \usebackgroundtemplate{}
 \color{gray!30!black}
   \begin{tikzpicture}[remember picture, overlay]
     \node[anchor = center, opacity=.25] (image) at (current page.center) {\includegraphics[scale=0.25]{chaospy_logo.jpg}};
   \end{tikzpicture}
 \begin{frame}[fragile,enviroment=chaospy]
   
}
{
 \end{frame}
}

 \lstset{
escapeinside={||},
basicstyle=\ttfamily\footnotesize, 
columns=fixed
}


\newenvironment{chaospy}[1]
{\color{gray!30!black}
     \color{gray!30!black}
     \usebackgroundtemplate{
   \begin{tikzpicture}[remember picture, overlay]
     \node[anchor = center, opacity=.25] (image) at (current page.center) {\includegraphics[scale=0.25]{chaospy_logo.jpg}};
   \end{tikzpicture}}
     \begin{frame}[fragile,environment=chaospy]
    \frametitle{{#1}}}
{\end{frame}}


\definecolor{keywords}{RGB}{255,0,90}
\definecolor{comments}{RGB}{0,0,113}
\definecolor{red}{RGB}{160,0,0}
\definecolor{green}{RGB}{0,150,0}
 
\usetheme{kalkulo}

\graphicspath{{./figures/}}


\title{Polynomial chaos expansions part 3: Intrusive Gallerkin method}
\author{Jonathan Feinberg and Simen Tennøe}


\begin{document}



\begin{frame}
  \maketitle
\end{frame}

\begin{frame}
 \frametitle{Repetition of our model problem}
  We have a simple differential equation
  \begin{align*}
    \frac{d u(x)}{dx} & =-au(x),\qquad u(0) = I
  \end{align*}
  \pause
  with the solution
  \[u(x) = Ie^{-ax}\]
  \pause
  with two random input variables:
   \[a \sim \text{Uniform(0, 0.1)}, \qquad I \sim \text{Uniform(8, 10)}\]
  Want to compute $\E{u}$ and $\Var{u}$
\end{frame}


\begin{frame}
 \frametitle{Calculating initial condition using intrusive Galerkin}
 \begin{align*}
 \hat u_M(0) &= I\\
  \onslide<2-> {\sum_{n=0}^Nc_n(0)P_n &= I}\\
  \onslide<3-> {\inner{\sum_{n=0}^Nc_n(0)P_n,P_k} &= \inner{ I,P_k}
  & k&=0,\dots,N}\\
  \onslide<4-> {\sum_{n=0}^Nc_n(0)\inner{ P_n,P_k} &= \inner{ I,P_k} }\\
  \onslide<5-> {c_k(0)\inner{ P_k, P_k} &= \inner{ I,P_k} }\\
  \onslide<6-> {c_k(0) &= \frac{\inner{I, P_k}}{\inner{P_k, P_k}} = \frac{E(IP_k)}{E(P_k^2)}}\\
   \end{align*}

\end{frame}


\begin{frame}
 \frametitle{Calculating main equation using intrusive Galerkin}
 \scriptsize
 \begin{align*}
  \frac{d}{dx}\left(\hat u_M \right) &= -a \hat u_M\\
  \onslide<2-> {\frac{d}{dx}\left(\sum_{n=0}^Nc_nP_n \right) &= -a \sum_{n=0}^Nc_nP_n}\\
 \onslide<3-> {\inner{ \frac{d}{dx}\left(\sum_{n=0}^Nc_nP_n
 \right),P_k} &= \inner{-a \sum_{n=0}^Nc_nP_n,P_k} &
 k=0,\dots,N}\\
 \onslide<4-> {\frac{d}{dx}\sum_{n=0}^Nc_n\inner{ P_n ,P_k} &= -\sum_{n=0}^Nc_n\inner{ aP_n,P_k}}\\
 \onslide<5-> {\frac d{dx} c_k \inner{P_k, P_k}
 &= -\sum_{n=0}^N c_n \inner{aP_n, P_k} }\\
 \onslide<6-> {\frac{d}{dx}c_k
 &= -\sum_{n=0}^N c_n \frac{\inner{aP_n,P_k}}{\inner{P_k, P_k}}
 = -\sum_{n=0}^N c_n \frac{\E{aP_nP_k}}{\E{P_k^2}}}
 \end{align*}

 
\end{frame}



\begin{frame}
 \frametitle{Projection results in a set of differential equations
 to solve}
 \begin{align*}
     \frac{d}{dx}c_k(x) &= -\sum_{n=0}^N c_n(x) \frac{E(aP_nP_k)}{E(P_k^2)}
     & k=0,\dots,N\\
 c_k(0) &= \frac{E(IP_k)}{E(P_k^2)}
 \end{align*}
 \end{frame}

 
\begin{frame}
 \frametitle{Projection structure contains many zero terms}
 \begin{columns}
     \column{.5\textwidth}
\begin{center}
    \begin{align*}
        E(P_nP_k)
    \end{align*}
  \includegraphics[width=\textwidth]{binary_matrix1.png}
\end{center}
     \column{.5\textwidth}
     \begin{center}
    \begin{align*}
        E(aP_nP_k)
    \end{align*}
  \includegraphics[width=\textwidth]{binary_matrix.png}
     \end{center}
 \end{columns}
 \end{frame}

\begin{chaospy}{Solving the set of differential equations
    numerically}
    \scriptsize
\begin{lstlisting}[language=python]
import chaospy as cp
import numpy as np
import odespy
|\pause|
dist_a = cp.Uniform(0, 0.1)
dist_I = cp.Uniform(8, 10)
dist = cp.J(dist_a, dist_I) # joint multivariate dist
|\pause|
P, norms = cp.orth_ttr(n, dist, retall=True)|\pause|
variable_a, variable_I = cp.variable(2)
\end{lstlisting}
\end{chaospy}

\begin{chaospy}{Solving the set of differential equations numerically}
    \scriptsize
    \pause
\begin{lstlisting}[language=python]
PP = cp.outer(P, P)
E_aPP = cp.E(variable_a*PP, dist)
E_IP = cp.E(variable_I*P, dist)
|\pause|
def advance_func(c, x):
  return -np.sum(c*E_aPP,-1)/norms
initial_condition = E_IP/norms
|\pause|
solver = odespy.RK4(advance_func)
solver.set_initial_condition(intitial_condition)
|\pause|
x = np.linspace(0, 10, 1000)
c_hat = solver.solve(x)[0]
|\pause|
u_hat = cp.sum(P*c_n, -1)
\end{lstlisting}
\end{chaospy}


\begin{frame}
 \frametitle{Within the scope of solvable problems, intrusive
 Galerkin usually converges faster}
  \begin{figure}
  %\caption{Binary matrix of $E(aP_nP_k)$}
  \includegraphics[width=0.85\textwidth]{convergence_gallerkin.png}
 \end{figure}
\end{frame}












\end{document}
